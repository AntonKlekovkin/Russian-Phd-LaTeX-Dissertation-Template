
{\actuality} 

В настоящее время активно проводятся исследования, направленные на создание робототехнических систем, использующих нетрадиционные способы перемещения в жидкости. К подобным робототехническим системам относят водные роботы, передвигающиеся посредством имитирования движения живых существ или благодаря наличию внутренних подвижных механизмов, обеспечивающих изменение положения центра масс или кинетического момента. 

%На данный момент для роботов, перемещающихся в жидкости, самым распространенным способом перемещения является движение за счет вращения гребных винтов, однако отдельный интерес представляет исследование динамики роботов, имитирующих способы перемещения живых существ или движущихся за счет внутренних механизмов. Создание подобных транспортных средств невозможно без проведения базовых исследований их динамических свойств и создания соответствующей теории управления. В связи с этим актуальным является проведение как теоретических, так и экспериментальных работ по созданию высокоманевренных мобильных управляемых систем, реализующих новые методы передвижения в жидкости.

Подводные и надводные роботы, управляемые внутренними механизмами, реализуют способ передвижения в жидкости, при котором форма оболочки робота остается неизменной и отсутствуют приводные элементы, которые взаимодействуют непосредственно с жидкостью или воздухом над ее поверхностью. Движение осуществляется за счет внутренних механизмов робота, которые могут изменять положение центра масс мобильного робота или внутренний гиростатический момент. Основными преимуществами данных роботов, перемещающихся в жидкости, являются отсутствие внешних подвижных элементов, простота конструкции, возможность полной гидроизоляции. В связи с этим данные роботы имеют преимущества в некоторых задачах перед другими роботами: работа на большой глубине с высоким давлением, работа в агрессивных средах или средах с высокой плотностью растительности.

%В настоящее время водные мобильные системы (основанные на традиционном способе перемещении с помощью винтов) широко используются для мониторинга и проведения различных операций в сложных условиях эксплуатации. В частности, для мониторинга подводного рельефа и подводной геологоразведки, мониторинга обшивок подводных конструкций, проведение ремонтных работ на больших глубинах, в условиях химического или радиационного загрязнения и т.д. Использование в аналогичных задачах безвинтовых транспортных средств имеет ряд конструкционных и эксплуатационных особенностей: изолированность рабочих узлов от жидкости, простота конструкции, повышенная экологичность.


Одни из первых теоретических исследований в данной области представлены в работах В. В. Козлова, С. М. Рамоданова, Д. А. Онищенко, в которых была показана возможность неограниченного продвижения тела в рамках модели идеальной жидкости при анизотропии присоединенных масс. В работах Ф. Л. Черноусько, Н. Н. Болотника рассмотрены модели движения с заданными законами сопротивления. С. Ф. Яцун с соавторами рассматривали различные математические модели плавающих роботов, приводимых в движение посредством периодических перемещений внутренних масс. Численное моделирование движения объекта с изменяемым центром масс в жидкости на основе совместного решения уравнений Навье-Стокса и уравнений динамики твердого тела проводилось В. А. Тененёвым, Е. В. Ветчаниным с соавторами. В работах А. А. Килина и А. И. Кленова рассмотрена локомоционная мобильная платформа, перемещающаяся по поверхности жидкости за счет изменения распределения масс. Из зарубежных ученых, работающих по данной тематике, можно выделить С. Чилдресса, Ф. Таллапрагаду, С. Д. Келли. %S. Childress, P. Tallapragada, S. D. Kelly. 
В своих исследованиях С. Чилдресс рассматривает влияние вязкости на самопродвижение твердого тела переменной формы с движущейся внутри него массой. Ф. Таллапрагада и С. Д. Келли были проведены одни из немногих экспериментальных исследований движения водного робота за счет вращения внутреннего ротора.  %Разработана математическая модель плоскопараллельного движения в рамках модели идеальной жидкости и математическая модель движения с учетом внешних сил, действующих на объект со стороны жидкости. Изготовлен натурный образец и проведены экспериментальные исследования.

Результаты исследований, полученные в данной области, подтверждают сложность динамики движения подобных систем, а также неочевидность процесса формирования управления для реализации движения вдоль простых траекторий. В связи с этим проведение теоретических и экспериментальных исследований водных роботов, использующих внутренние механизмы для продвижения в жидкости, является актуальным.

В данной работе проведены исследования двух объектов: безвинтового подводного робота с осесимметричной оболочкой и безвинтового надводного робота с оболочкой, имеющей форму симметричного профиля с острой кромкой. Оба робота передвигаются за счет изменения гиростатического момента, возникающего за счет вращения роторов, расположенных внутри оболочки.
 
Безвинтовой подводный робот с осесимметричной оболочкой может двигаться как при частичном, так и при полном погружении. Для описания его движения разработана трехмерная математическая модель в рамках теории идеальной жидкости. Безвинтовой надводный робот с оболочкой, имеющей форму симметричного профиля с острой кромкой движется по поверхности жидкости. Разработана математическая модель, описывающая плоскопараллельное движение робота и учитывающая сопротивление жидкости. На основе математических моделей сформированы режимы управления роторами для различных типов движения для каждого из роботов, проведены экспериментальные исследования.




{\aim} данной работы является исследование принципов движения роботов в жидкости, управляемых внутренними механизмами, и разработка алгоритмов управления их движением.
%динамики механизмов, обеспечивающих движение водных роботов за счет изменения гиростатического момента, возникающего при вращении роторов, расположенных внутри роботов.

Для~достижения поставленной цели необходимо решить следующие {\tasks}:
\begin{enumerate}
  \item Построение и исследование математической модели движения безвинтового подводного робота в жидкости с механизмами, обеспечивающими создание внутреннего гиростатического момента.
  \item Разработка алгоритма управления движением безвинтового подводного робота в жидкости на базе предложенной математической модели.
  \item Построение и исследование математической модели движения безвинтового надводного робота, перемещающегося по поверхности жидкости за счет изменения внутреннего гиростатического момента.
  \item Разработка алгоритма управления движением безвинтового надводного робота по поверхности жидкости на базе предложенной математической модели.
  \item Анализ и синтез механизмов, обеспечивающих изменение внутреннего гиростатического момента и разработка конструкции прототипов водных роботов: безвинтовых подводного и надводного роботов; разработка систем управления.
  \item %Создание натурных образцов и методик оценки характеристик их движения по поверхности и в толще жидкости.
  Создание натурных образцов и методик экспериментальной оценки характеристик их движения в жидкости. %Проведение натурных экспериментов и исследование влияния режимов работы механизма на динамику роботов: безвинтового подводного робота в форме эллипсоида и недеформируемого водного робота с острой кромкой.
  \item Проведение экспериментальных исследований и сравнение полученных данных с результатами численного моделирования для оценки разработанных алгоритмов управления.
\end{enumerate}


{\novelty} заключается в разработанных математических моделях движения роботов в жидкости, управляемых внутренними механизмами, в алгоритмах управления, построенных на базе предложенных математических моделей, и результатах их экспериментальной апробации.
%\begin{enumerate}
%  %  \item Разработаны основы структурного синтеза механизма, осуществляющего изменение распределения масс для локомоционного водного робота.
%  \item Разработана математическая модель движения безвинтового подводного робота в форме эллипсоида в жидкости за счет изменения внутреннего гиростатического момента.
%  \item Разработана математическая модель движения недеформируемого водного робота с острой кромкой по поверхности жидкости при частичном погружении за счет изменения внутреннего гиростатического момента с учетом вязкого сопротивления жидкости.
%  %\item \todo{Разработан оригинальный алгоритм управления безвинтовым подводным роботом в форме эллипсоида в жидкости за счет изменения внутреннего гиростатического момента.}
%  \item Разработан алгоритм управления недеформируемым водным роботом с острой кромой в жидкости за счет изменения внутреннего гиростатического момента.
%  \item Разработаны конструкции мобильных водных роботов, перемещающихся за счет изменения внутреннего гиростатического момента: безвинтового подводного робота в форме эллипсоида и недеформируемого водного робота с острой кромкой.
%  \item Получены результаты экспериментальных исследований, на основе которых сделана оценка возможности использования разработанных математических моделей движения и алгоритмов управления для безвинтового подводного робота в форме эллипсоида и недеформируемого водного робота с острой кромкой.
%\end{enumerate}

{\influence}. Результаты, полученные в диссертации, могут быть использованы для разработки новых или усовершенствования существующих водных мобильных аппаратов. Полученные теоретические модели движения могут использоваться для вычисления оптимальных конструкционных параметров механизмов мобильных роботов, перемещающихся в жидкости. Также разработанные математические модели позволяют определить управляющие воздействия для элементарных маневров, которые можно комбинировать и использовать при перемещении роботов подобной конструкции по сложной траектории. Таким образом работа с рассматриваемыми роботами позволяет проводить полноценные исследования движения мобильных водных роботов, что делает их наглядным лабораторным комплексом, который можно использовать в учебном процессе.

{\methods} Для решения поставленных в рамках диссертационного исследования задач использовались аналитические и численные методы решения уравнений динамики. Для вычисления коэффициентов присоединенных масс и коэффициентов вязкого сопротивления в математической модели движения безвинтового надводного робота с острой кромкой использовался метод численного решения уравнений Навье-Стокса. При проведении экспериментальных исследований движения роботов использовалась система захвата движения. Обработка данных, полученных из экспериментов, проводилась с использованием программных комплексов Matlab, Maple. Программное обеспечение управления роботами для микроконтроллеров разрабатывалось на языке программирования Си в среде Keil uVision.

{\defpositions}
\begin{enumerate}
  \item Математическая модель движения безвинтового подводного робота в жидкости за счет изменения внутреннего гиростатического момента.
  \item Математическая модель движения безвинтового надводного робота, перемещающегося по поверхности жидкости за счет изменения внутреннего гиростатического момента, с учетом вязкого сопротивления среды.
  %\item Алгоритм управления безвинтовым подводным роботом в форме эллипсоида в жидкости за счет изменения внутреннего гиростатического момента.
  \item Алгоритм управления движением безвинтового надводного робота по поверхности жидкости за счет изменения внутреннего гиростатического момента.
  \item Конструкции безвинтовых подводного и надводного роботов, реализующих движение в жидкости за счет изменения внутреннего гиростатического момента.
  \item Результаты экспериментальных исследований по оценке разработанных алгоритмов управления для безвинтовых подводного и надводного роботов.
  
\end{enumerate}
%В папке Documents можно ознакомиться в решением совета из Томского ГУ в~файле \verb+Def_positions.pdf+, где обоснованно даются рекомендации по~формулировкам защищаемых положений.

\textbf{\underline{Область исследования}.} Диссертационная работа выполнена в соответствии с паспортом специальности ВАК 05.02.05~--- <<Роботы, мехатроника и робототехнические системы>> по пунктам: 

1. Методы анализа и оптимизационного синтеза роботов, робототехнических и мехатронных систем. 

2. Математическое моделирование мехатронных и робототехнических систем, анализ их характеристик методами компьютерного моделирования, разработка новых методов управления и проектирования таких систем.

{\reliability} Разработанные математические модели основаны на классических утверждениях и теоремах и не противоречат известным результатам. Для решения и исследования полученных уравнений применялись апробированные аналитические и численные методы. Достоверность подтверждается согласованностью математической модели с результатами натурных экспериментов. Для проведения экспериментальных исследований использовались современные измерительные комплексы.


{\probation}
Основные результаты работы обсуждались на семинарах кафедры «Мехатронные системы» ФГБОУ ВПО «Ижевский государственный технический университет имени М.Т. Калашникова», «Института компьютерных исследований» ФГБОУ ВПО «Удмуртский государственный университет», Центра технологий компонентов робототехники и мехатроники Университета Иннополис.

Кроме того, результаты исследований, изложенные в диссертации, докладывались на российских и международных конференциях:
Международная конференция «GDIS-2016» (Ижевск, 2016 г.), Международная конференция МИКМУС-2018 (Москва, 2018 г.), Международная конференция "Scientific Heritage of Sergey A. Chaplygin: nonholonomic mechanics, vortex structures and hydrodynamics" (Чебоксары, 2019 г.), Международная конференция "Экстремальная робототехника-2019" (Санкт-Петербург, 2019 г.), Международная конференция CLAWAR-2020 (Москва, 2020 г.).

%По результатам диссертационного исследования получены авторские права на следующие результаты интеллектуальной деятельности:
%\begin{enumerate}
%	
%	\item Патент на полезную модель. №172254 РФ. Безвинтовой подводный робот //  А.В. Борисов, И.С. Мамаев, А.А. Килин, А.А. Калинкин, Ю.Л. Караваев, А.В. Клековкин, Е.В. Ветчанин. Заявка: 2016144812, 15.11.2016, опубл. 3.07.2017
%	
%	\item № 2017613219. Программа для управления безвинтовым подводным роботом // А.В. Борисов, И.С. Мамаев, А.А. Килин, Ю.Л. Караваев, А.В. Клековкин. Заявка: 2016662663, 22.11.2016, опубл. 16.03.2017
%	
%	\item № 2019612284. Программа управления безвинтовым надводным роботом с внутренним ротором // А.В. Борисов, И.С. Мамаев, А.А. Килин, А.В. Клековкин, Ю.Л. Караваев. Заявка: 2019610925, 04.02.2019, опубл. 14.02.2019
%	
%\end{enumerate}

{\contribution} Постановки задач, обсуждение результатов проводились совместно с руководителем и соавторами работ. Соискателем разработаны математические модели, прототипы мобильных платформ, программное обеспечение для управления мобильными роботами; проведены численные и натурные эксперименты, проведена обработка результатов экспериментов.

{\publications} Основные результаты по теме диссертации изложены в 11 публикациях, 3 из которых изданы в журналах, рекомендованных ВАК, из них 2 публикации изданы в научных журналах, индексируемых Web of Science и Scopus. Одна публикация издана в сборниках докладов конференций, индексируемых Scopus, 4 — в сборниках докладов и тезисах конференций. Получен патент на полезную модель и 2 свидетельства о регистрации программы для ЭВМ.

%\ifnumequal{\value{bibliosel}}{0}
%{%%% Встроенная реализация с загрузкой файла через движок bibtex8. (При желании, внутри можно использовать обычные ссылки, наподобие `\cite{vakbib1,vakbib2}`).
%    {\publications} Основные результаты по теме диссертации изложены в XX печатных изданиях,
%    X из которых изданы в журналах, рекомендованных ВАК,
%    X "--- в сборниках докладов и тезисах конференций.
%}%
%{%%% Реализация пакетом biblatex через движок biber
%    \begin{refsection}[bl-author]
%        % Это refsection=1.
%        % Процитированные здесь работы:
%        %  * подсчитываются, для автоматического составления фразы "Основные результаты ..."
%        %  * попадают в авторскую библиографию, при usefootcite==0 и стиле `\insertbiblioauthor` или `\insertbiblioauthorgrouped`
%        %  * нумеруются там в зависимости от порядка команд `\printbibliography` в этом разделе.
%        %  * при использовании `\insertbiblioauthorgrouped`, порядок команд `\printbibliography` в нём должен быть тем же (см. biblio/biblatex.tex)
%        %
%        % Невидимый библиографический список для подсчёта количества публикаций:
%        \printbibliography[heading=nobibheading, section=1, env=countauthorvak,          keyword=biblioauthorvak]%
%        \printbibliography[heading=nobibheading, section=1, env=countauthorwos,          keyword=biblioauthorwos]%
%        \printbibliography[heading=nobibheading, section=1, env=countauthorscopus,       keyword=biblioauthorscopus]%
%        \printbibliography[heading=nobibheading, section=1, env=countauthorconf,         keyword=biblioauthorconf]%
%        \printbibliography[heading=nobibheading, section=1, env=countauthorother,        keyword=biblioauthorother]%
%        \printbibliography[heading=nobibheading, section=1, env=countauthor,             keyword=biblioauthor]%
%        \printbibliography[heading=nobibheading, section=1, env=countauthorvakscopuswos, filter=vakscopuswos]%
%        \printbibliography[heading=nobibheading, section=1, env=countauthorscopuswos,    filter=scopuswos]%
%        %
%        \nocite{*}%
%        %
%        {\publications} Основные результаты по теме диссертации изложены в~\arabic{citeauthor}~печатных изданиях,
%        \arabic{citeauthorvak} из которых изданы в журналах, рекомендованных ВАК\sloppy%
%        \ifnum \value{citeauthorscopuswos}>0%
%            , \arabic{citeauthorscopuswos} "--- в~периодических научных журналах, индексируемых Web of~Science и Scopus\sloppy%
%        \fi%
%        \ifnum \value{citeauthorconf}>0%
%            , \arabic{citeauthorconf} "--- в~тезисах докладов.
%        \else%
%            .
%        \fi
%    \end{refsection}%
%    \begin{refsection}[bl-author]
%        % Это refsection=2.
%        % Процитированные здесь работы:
%        %  * попадают в авторскую библиографию, при usefootcite==0 и стиле `\insertbiblioauthorimportant`.
%        %  * ни на что не влияют в противном случае
%        \nocite{vakbib2}%vak
%        \nocite{bib1}%other
%        \nocite{confbib1}%conf
%    \end{refsection}%
%        %
%        % Всё, что вне этих двух refsection, это refsection=0,
%        %  * для диссертации - это нормальные ссылки, попадающие в обычную библиографию
%        %  * для автореферата:
%        %     * при usefootcite==0, ссылка корректно сработает только для источника из `external.bib`. Для своих работ --- напечатает "[0]" (и даже Warning не вылезет).
%        %     * при usefootcite==1, ссылка сработает нормально. В авторской библиографии будут только процитированные в refsection=0 работы.
%        %
%        % Невидимый библиографический список для подсчёта количества внешних публикаций
%        % Используется, чтобы убрать приставку "А" у работ автора, если в автореферате нет
%        % цитирований внешних источников.
%        % Замедляет компиляцию
%    \ifsynopsis
%    \ifnumequal{\value{draft}}{0}{
%      \printbibliography[heading=nobibheading, section=0, env=countexternal,          keyword=biblioexternal]%
%    }{}
%    \fi
%}

%При использовании пакета \verb!biblatex! будут подсчитаны все работы, добавленные
%в файл \verb!biblio/author.bib!. Для правильного подсчёта работ в~различных
%системах цитирования требуется использовать поля:
%\begin{itemize}
%        \item \texttt{authorvak} если публикация индексирована ВАК,
%        \item \texttt{authorscopus} если публикация индексирована Scopus,
%        \item \texttt{authorwos} если публикация индексирована Web of Science,
%        \item \texttt{authorconf} для докладов конференций,
%        \item \texttt{authorother} для других публикаций.
%\end{itemize}
%Для подсчёта используются счётчики:
%\begin{itemize}
%        \item \texttt{citeauthorvak} для работ, индексируемых ВАК,
%        \item \texttt{citeauthorscopus} для работ, индексируемых Scopus,
%        \item \texttt{citeauthorwos} для работ, индексируемых Web of Science,
%        \item \texttt{citeauthorvakscopuswos} для работ, индексируемых одной из трёх баз,
%        \item \texttt{citeauthorscopuswos} для работ, индексируемых Scopus или Web of~Science,
%        \item \texttt{citeauthorconf} для докладов на конференциях,
%        \item \texttt{citeauthorother} для остальных работ,
%        \item \texttt{citeauthor} для суммарного количества работ.
%\end{itemize}
%% Счётчик \texttt{citeexternal} используется для подсчёта процитированных публикаций.
%
%Для добавления в список публикаций автора работ, которые не были процитированы в
%автореферате требуется их~перечислить с использованием команды \verb!\nocite! в
%\verb!Synopsis/content.tex!.
