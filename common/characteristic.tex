
{\actuality} 

В настоящее время, в мире ведутся активные исследования по созданию новых способов передвижения. Ведутся разработки «экзотических» транспортных средств, перспективность применения которых связана с их использованием в специфических (критических) условиях, например в условиях космоса или других планет. Максимальная степень защиты внутренних функциональных элементов и внешних рабочих органов мобильных роботов существенно повышает сложность конструкции, массо-габаритные характеристики, что негативно сказывается на маневренности, производительности и точности позиционирования мобильных робототехнических систем при реализации алгоритмов управления. Таким образом, актуальным является создание транспортных средств, в которых изначально конструкционно заложена высокая степень защиты внутреннего пространства благодаря минимальному количеству внешних рабочих органов.

Создание подобных транспортных средств невозможно без проведения базовых исследований их динамических свойств и создания соответствующей теории управления. В связи с этим актуальным является проведение как теоретических, так и прикладных работ по созданию высокоманевренных мобильных управляемых систем, реализующих новые методы передвижения в жидкости.

Безвинтовые подводные и надводные роботы – это принципиально новое направление, как в науке, так и технике. Данные роботы реализуют нетрадиционный способ передвижения в жидкости, при котором форма оболочки робота остается неизменной и отсутствуют приводные элементы, которые взаимодействуют непосредственно с жидкостью или воздухом над ее поверхностью. Движение осуществляется за счет внутренних механизмов робота, которые могут изменять положение центра масс мобильного робота или гиростатический момент. 

В настоящее время водные мобильные системы (но основанные на традиционном способе перемещении с помощью винтов) широко используются для мониторинга и проведения различных операций в сложных условиях эксплуатации. В частности, для мониторинга подводного рельефа и подводной геологоразведки, мониторинга обшивок подводных конструкций, проведение ремонтных работ на больших глубинах и в условиях химического или радиационного загрязнения и т.д. Использование в аналогичных приложениях безвинтовых транспортных средств имеет ряд конструкционных и эксплуатационных преимуществ: изолированность рабочих узлов от жидкости, простота конструкции, повышенная маневренность, гидроакустическая бесшумность, повышенная экологичность.

Одни из первых теоретических исследований в данной области представлены в работах академика РАН В.В. Козлова, доктора наук С.М. Рамоданова. Так же можно выделить работы академика РАН Ф.Л. Черноусько, докторов наук Д.А. Онищенко, С.Ф. Яцуна, В.А. Тененёва. В этих работах проведено численное моделирование движения объекта с изменяемым центром масс в жидкости. Работы с натурными экспериментами практически отсутствуют.

В диссертации~\cite{Klenov_diss} рассмотрена локомоционная мобильная платформа, перемещающаяся в жидкости за счет изменения распределения масс. Разработана математическая модель плоскопараллельного движения для идеальной жидкости и математическая модель движения с учетом внешних сил, действующих на объект со стороны жидкости. Изготовлен натурный образец и проведены экспериментальные исследования, при этом платформа движется по поверхности жидкости.

Данная работа является продолжением исследований, выполенных в~\cite{Klenov_diss2}. Рассмотрено два объекта: безвинтовой мобильный робот в форме эллипсоида и недеформируемый робоподобный робот. Оба объекта передвигаются за счет изменения гиростатического момента, возникающего за счет вращения роторов, расположенных внутри оболочки роботов.
 
Безвинтовой мобильный робот в форме эллипсоида может двигаться как по поверхности жидкости так и в толще воды. Для описания его движения разработана трехмерная математическая модель для идеальной жидкости. На основе этой модели подобраны режимы управления роторами для различного движения. Проведены экспериментальные исследования.

Недеформируемый робоподобный робот движется по поверхности жидкости. Разработана математическая модель описывающая плоскопараллельное движение робота, учитывающая сопротивление жидкости и циркуляцию. На основе этой модели подобраны режимы управления роторами для различного движения. Проведены экспериментальные исследования.


%В связи с изложенным выше, тема диссертационной работы представляется актуальной.



%Обзор, введение в тему, обозначение места данной работы в
%мировых исследованиях и~т.\:п., можно использовать ссылки на~другие
%работы~\autocite{Gosele1999161}
%(если их~нет, то~в~автореферате
%автоматически пропадёт раздел <<Список литературы>>). Внимание! Ссылки
%на~другие работы в~разделе общей характеристики работы можно
%использовать только при использовании \verb!biblatex! (из-за технических
%ограничений \verb!bibtex8!. Это связано с тем, что одна
%и~та~же~характеристика используются и~в~тексте диссертации, и в
%автореферате. В~последнем, согласно ГОСТ, должен присутствовать список
%работ автора по~теме диссертации, а~\verb!bibtex8! не~умеет выводить в одном
%файле два списка литературы).
%При использовании \verb!biblatex! возможно использование исключительно
%в~автореферате подстрочных ссылок
%для других работ командой \verb!\autocite!, а~также цитирование
%собственных работ командой \verb!\cite!. Для этого в~файле
%\verb!common/setup.tex! необходимо присвоить положительное значение
%счётчику \verb!\setcounter{usefootcite}{1}!.
%
%Для генерации содержимого титульного листа автореферата, диссертации
%и~презентации используются данные из файла \verb!common/data.tex!. Если,
%например, вы меняете название диссертации, то оно автоматически
%появится в~итоговых файлах после очередного запуска \LaTeX. Согласно
%ГОСТ 7.0.11-2011 <<5.1.1 Титульный лист является первой страницей
%диссертации, служит источником информации, необходимой для обработки и
%поиска документа>>. Наличие логотипа организации на~титульном листе
%упрощает обработку и~поиск, для этого разметите логотип вашей
%организации в папке images в~формате PDF (лучше найти его в векторном
%варианте, чтобы он хорошо смотрелся при печати) под именем
%\verb!logo.pdf!. Настроить размер изображения с логотипом можно
%в~соответствующих местах файлов \verb!title.tex!  отдельно для
%диссертации и автореферата. Если вам логотип не~нужен, то просто
%удалите файл с~логотипом.
%
%\ifsynopsis
%Этот абзац появляется только в~автореферате.
%Для формирования блоков, которые будут обрабатываться только в~автореферате,
%заведена проверка условия \verb!\!\verb!ifsynopsis!.
%Значение условия задаётся в~основном файле документа (\verb!synopsis.tex! для
%автореферата).
%\else
%Этот абзац появляется только в~диссертации.
%Через проверку условия \verb!\!\verb!ifsynopsis!, задаваемого в~основном файле
%документа (\verb!dissertation.tex! для диссертации), можно сделать новую
%команду, обеспечивающую появление цитаты в~диссертации, но~не~в~автореферате.
%\fi

% {\progress}
% Этот раздел должен быть отдельным структурным элементом по
% ГОСТ, но он, как правило, включается в описание актуальности
% темы. Нужен он отдельным структурынм элемементом или нет ---
% смотрите другие диссертации вашего совета, скорее всего не нужен.

{\aim} данной работы является исследование механизмов, обеспечивающих движение водоплавающих роботов за счет изменения гиростатического момента, возникающего от вращения роторов, расположенных внутри оболочки объектов.

Для~достижения поставленной цели необходимо решить следующие {\tasks}:
\begin{enumerate}
  \item Построение математической модели движения мобильного робота в форме эллипсоида в жидкости за счет изменения внутреннего кинетического момента.
  \item Построение математической модели движения недеформируемого робоподобного робота в жидкости за счет изменения внутреннего кинетического момента с учетом вязкого трения и циркуляции.
  \item Разработка алгоритмов управления для реализации движения мобильных водоплавающих роботов.
  \item Разработка прототипов роботов: разработка конструкции мобильных роботов; разработка систем управления.
  \item Проведение натурных экспериментов и исследования влияния режимов работы механизма на динамику роботов.
  \item Сравнение экспериментальных данных с результатами численного моделирования.
\end{enumerate}


{\novelty}
\begin{enumerate}
  %  \item Разработаны основы структурного синтеза механизма, осуществляющего изменение распределения масс для локомоционного водного робота.
  \item Разработана оригинальная математическая модель движения мобильного робота в форме эллипсоида в жидкости за счет изменения внутреннего кинетического момента.
  \item Разработана оригинальная математическая модель движения недеформируемого робоподобного робота в жидкости за счет изменения внутреннего кинетического момента с учетом вязкого трения и циркуляции.
  \item Разработан оригинальный алгоритм управления мобильным роботом в форме эллипсоида в жидкости за счет изменения внутреннего кинетического момента.
  \item Разработан оригинальный алгоритм управления недеформируемым робоподобным роботом в жидкости за счет изменения внутреннего кинетического момента.
  \item Разработаны оригинальные конструкции мобильных водоплавающих роботов, перемещающихся за счет изменения внутреннего кинетического момента: робота в форме эллипсоида и недеформируемого рыбоподобного робота.
  \item Получены результаты экспериментальных исследований по оценке разработанных алгоритмов управления.
\end{enumerate}

{\influence} Результаты, изложенные в диссертации, могут быть использованы для проектирования (усовершенствования) мобильных устройств перемещающихся в жидкости. Разработанные математические модели движения могут использоваться для определения оптимальных параметров механизмов подобных роботов, перемещающихся в жидкости и построения систем управления. Разработанная методика определения гидродинамических сил позволяет определять присоединенные массы и коэффициенты гидродинамического сопротивления тел, движущихся в жидкости.

Безвинтовые плавающие роботы с вращающимися внутренними роторами являются примером сложных динамических систем, на основе которых можно проводить как моделирование, так и экспериментальные исследования, дополняя или упрощая существующие конструкции, что делает их наглядным лабораторным комплексом, который можно внедрять в учебный процесс.

{\methods} Для решения поставленных в рамках диссертационного исследования задач использовались методы теории машин и механизмов, аналитические и численные методы решения уравнений динамики. При проведении экспериментальных исследований движения роботов использовались современные технологии захвата движения: для отслеживания движения подводного робота использовалась система Motion Capture, состоящая из 4 камер, предназначенная для работы под водой; для отслеживания движения на поверхности жидкости использовалась система Motion Capture, состоящая из 7 камер. Обработка результатов экспериментов проводилась с использованием программных комплексов Matlab, Maple. Программное обеспечение для управления роботом разрабатывалось на языке Си для микроконтроллеров серии STM32F303 с ядром Cortex-M4 в среде Keil uVision 5.

{\defpositions}
\begin{enumerate}
  \item Математическая модель движения мобильного робота в форме эллипсоида в жидкости за счет изменения внутреннего кинетического момента.
  \item Математическая модель движения недеформируемого робоподобного робота в жидкости за счет изменения внутреннего кинетического момента с учетом вязкого трения и циркуляции.
  \item Алгоритм управления мобильным роботом в форме эллипсоида в жидкости за счет изменения внутреннего кинетического момента.
  \item Алгоритм управления недеформируемым робоподобным роботом в жидкости за счет изменения внутреннего кинетического момента.
  \item Результаты экспериментальных исследований по оценке разработанных алгоритмов управления
  
\end{enumerate}
%В папке Documents можно ознакомиться в решением совета из Томского ГУ в~файле \verb+Def_positions.pdf+, где обоснованно даются рекомендации по~формулировкам защищаемых положений.

{\reliability} Разработанные математические модели основываются на классических утверждениях и теоремах и не противоречат известным результатам. Для исследования и моделирования полученных уравнений используются апробированные аналитические и численные методы решения. Достоверность подтверждается согласованностью математической модели с результатами натурных экспериментов. Для проведения экспериментальных исследований использовались современные измерительные комплексы, прошедшие поверку.


{\probation}
Основные результаты работы обсуждались на семинарах «Института компьютерных исследований» ФГБОУ ВПО «Удмуртский государственный университет», кафедры «Мехатронные системы» ФГБОУ ВПО «Ижевский государственный технический университет имени М.Т. Калашникова».

Кроме того, результаты исследований, изложенные в диссертации, докладывались на российских и международных конференциях:
\begin{itemize}
	\item IV Всероссийская научно-техническая конференция аспирантов, магистрантов и молодых ученых с международным участием «Молодые ученые -- ускорению научно-технического прогресса в XXI веке». (Ижевск, 2016).
	\item Шестая международная конференция «Geometry, Dynamics, Integrable Systems -- GDIS 2016» (Ижевск, 2016 г.)
	\item Машиноведение и инновации. Конференция молодых учёных и студентов (МИКМУС-2018) (Москва, 2018 г.)
	\item International Conference "Scientific Heritage of Sergey A. Chaplygin: nonholonomic mechanics, vortex structures and hydrodynamics" (Чебоксары, 2019 г.)
	\item 30-я международная научно-техническая конференция "Экстремальная робототехника-2019" (Санкт-Петербург, 2019 г.)	
\end{itemize}

По результатам диссертационного исследования получены авторские права на следующие результаты интеллектуальной деятельности:
\begin{enumerate}
	
	\item Патент на полезную модель. №172254 РФ. Безвинтовой подводный робот //  А.В. Борисов, И.С. Мамаев, А.А. Килин, А.А. Калинкин, Ю.Л. Караваев, А.В. Клековкин, Е.В. Ветчанин. Заявка: 2016144812, 15.11.2016, опубл. 3.07.2017
	
	\item № 2017613219. Программа для управления безвинтовым подводным роботом // А.В. Борисов, И.С. Мамаев, А.А. Килин, Ю.Л. Караваев, А.В. Клековкин. Заявка: 2016662663, 22.11.2016, опубл. 16.03.2017
	
	\item № 2019612284. Программа управления безвинтовым надводным роботом с внутренним ротором // А.В. Борисов, И.С. Мамаев, А.А. Килин, А.В. Клековкин, Ю.Л. Караваев. Заявка: 2019610925, 04.02.2019, опубл. 14.02.2019
	
\end{enumerate}

{\contribution} Постановки задач, обсуждение результатов поводились совместно с руководителем и соавторами работ. Автором разработаны математические модели, прототипы мобильных платформ, программное обеспечение для управления и мобильными роботами и обработки результатов эксперимента, проведены натурные и численные эксперименты.

\ifnumequal{\value{bibliosel}}{0}
{%%% Встроенная реализация с загрузкой файла через движок bibtex8. (При желании, внутри можно использовать обычные ссылки, наподобие `\cite{vakbib1,vakbib2}`).
    {\publications} Основные результаты по теме диссертации изложены в XX печатных изданиях,
    X из которых изданы в журналах, рекомендованных ВАК,
    X "--- в тезисах докладов.
}%
{%%% Реализация пакетом biblatex через движок biber
    \begin{refsection}[bl-author]
        % Это refsection=1.
        % Процитированные здесь работы:
        %  * подсчитываются, для автоматического составления фразы "Основные результаты ..."
        %  * попадают в авторскую библиографию, при usefootcite==0 и стиле `\insertbiblioauthor` или `\insertbiblioauthorgrouped`
        %  * нумеруются там в зависимости от порядка команд `\printbibliography` в этом разделе.
        %  * при использовании `\insertbiblioauthorgrouped`, порядок команд `\printbibliography` в нём должен быть тем же (см. biblio/biblatex.tex)
        %
        % Невидимый библиографический список для подсчёта количества публикаций:
        \printbibliography[heading=nobibheading, section=1, env=countauthorvak,          keyword=biblioauthorvak]%
        \printbibliography[heading=nobibheading, section=1, env=countauthorwos,          keyword=biblioauthorwos]%
        \printbibliography[heading=nobibheading, section=1, env=countauthorscopus,       keyword=biblioauthorscopus]%
        \printbibliography[heading=nobibheading, section=1, env=countauthorconf,         keyword=biblioauthorconf]%
        \printbibliography[heading=nobibheading, section=1, env=countauthorother,        keyword=biblioauthorother]%
        \printbibliography[heading=nobibheading, section=1, env=countauthor,             keyword=biblioauthor]%
        \printbibliography[heading=nobibheading, section=1, env=countauthorvakscopuswos, filter=vakscopuswos]%
        \printbibliography[heading=nobibheading, section=1, env=countauthorscopuswos,    filter=scopuswos]%
        %
        \nocite{*}%
        %
        {\publications} Основные результаты по теме диссертации изложены в~\arabic{citeauthor}~печатных изданиях,
        \arabic{citeauthorvak} из которых изданы в журналах, рекомендованных ВАК\sloppy%
        \ifnum \value{citeauthorscopuswos}>0%
            , \arabic{citeauthorscopuswos} "--- в~периодических научных журналах, индексируемых Web of~Science и Scopus\sloppy%
        \fi%
        \ifnum \value{citeauthorconf}>0%
            , \arabic{citeauthorconf} "--- в~тезисах докладов.
        \else%
            .
        \fi
    \end{refsection}%
    \begin{refsection}[bl-author]
        % Это refsection=2.
        % Процитированные здесь работы:
        %  * попадают в авторскую библиографию, при usefootcite==0 и стиле `\insertbiblioauthorimportant`.
        %  * ни на что не влияют в противном случае
        \nocite{vakbib2}%vak
        \nocite{bib1}%other
        \nocite{confbib1}%conf
    \end{refsection}%
        %
        % Всё, что вне этих двух refsection, это refsection=0,
        %  * для диссертации - это нормальные ссылки, попадающие в обычную библиографию
        %  * для автореферата:
        %     * при usefootcite==0, ссылка корректно сработает только для источника из `external.bib`. Для своих работ --- напечатает "[0]" (и даже Warning не вылезет).
        %     * при usefootcite==1, ссылка сработает нормально. В авторской библиографии будут только процитированные в refsection=0 работы.
        %
        % Невидимый библиографический список для подсчёта количества внешних публикаций
        % Используется, чтобы убрать приставку "А" у работ автора, если в автореферате нет
        % цитирований внешних источников.
        % Замедляет компиляцию
    \ifsynopsis
    \ifnumequal{\value{draft}}{0}{
      \printbibliography[heading=nobibheading, section=0, env=countexternal,          keyword=biblioexternal]%
    }{}
    \fi
}

%При использовании пакета \verb!biblatex! будут подсчитаны все работы, добавленные
%в файл \verb!biblio/author.bib!. Для правильного подсчёта работ в~различных
%системах цитирования требуется использовать поля:
%\begin{itemize}
%        \item \texttt{authorvak} если публикация индексирована ВАК,
%        \item \texttt{authorscopus} если публикация индексирована Scopus,
%        \item \texttt{authorwos} если публикация индексирована Web of Science,
%        \item \texttt{authorconf} для докладов конференций,
%        \item \texttt{authorother} для других публикаций.
%\end{itemize}
%Для подсчёта используются счётчики:
%\begin{itemize}
%        \item \texttt{citeauthorvak} для работ, индексируемых ВАК,
%        \item \texttt{citeauthorscopus} для работ, индексируемых Scopus,
%        \item \texttt{citeauthorwos} для работ, индексируемых Web of Science,
%        \item \texttt{citeauthorvakscopuswos} для работ, индексируемых одной из трёх баз,
%        \item \texttt{citeauthorscopuswos} для работ, индексируемых Scopus или Web of~Science,
%        \item \texttt{citeauthorconf} для докладов на конференциях,
%        \item \texttt{citeauthorother} для остальных работ,
%        \item \texttt{citeauthor} для суммарного количества работ.
%\end{itemize}
%% Счётчик \texttt{citeexternal} используется для подсчёта процитированных публикаций.
%
%Для добавления в список публикаций автора работ, которые не были процитированы в
%автореферате требуется их~перечислить с использованием команды \verb!\nocite! в
%\verb!Synopsis/content.tex!.
