%% Согласно ГОСТ Р 7.0.11-2011:
%% 5.3.3 В заключении диссертации излагают итоги выполненного исследования, рекомендации, перспективы дальнейшей разработки темы.
%% 9.2.3 В заключении автореферата диссертации излагают итоги данного исследования, рекомендации и перспективы дальнейшей разработки темы.
\begin{enumerate}
  \item Разработана математическая модель движения безвинтового подводного робота в форме эллипсоида в жидкости за счет внутреннего гиростатического момента в рамках теории идеальной жидкости. Исследования управляемости модели показали, что для полной управляемости форму робота необходимо сделать винтовой.
  
  \item Разработаны конструкция и экспериментальный образец безвинтового подводного робота в форме эллипсоида. Учтено требование в необходимости винтовой формы робота, к оболочке в виде эллипсоида вращения добавлены винтовые лопасти. Проработана конструкция внутренних компонентов с учетом ограничений в размерах. Разработана система управления роботом.
  
  \item Проведены экспериментальные исследования безвинтового подводного робота в форме эллипсоида, проанализированы результаты. Для более точного описания движения необходимо учитывать в модели вязкое сопротивление, генерацию вихревых структур. В рамках дальнейшего исследования рассматривается движение объекта в форме симметричного профиля NACA 0040 по поверхности жидкости.
  
  \item Разработана математическая модель движения безвинтового надводного робота с оболочкой, имеющей форму симметричного профиля NACA 0040 и острую кромку, в жидкости за счет внутреннего гиростатического момента с учетом вязкого сопротивления. На основе анализа математической модели разработан алгоритм управления роботом, определена форма управляющего воздействия для движения вдоль прямой и вдоль окружности.
  
  \item Разработаны конструкция и экспериментальный образец безвинтового надводного робота с острой кромкой. Разработана система управления роботом.
  
  \item Проведены экспериментальные исследования с безвинтовым надводным роботом с острой кромкой, проанализированы результаты. Показано, что разработанная математическая модель качественно описывает движение робота. Рассмотрены движения вдоль прямой и вдоль окружности при различных управляющих воздействиях.
  
  \item По разработанным конструкциям получен патент на полезную модель безвинтового подводного робота в форме эллипсоида, для разработанных программных продуктов получены свидетельства о регистрации программ ЭВМ.
\end{enumerate}
