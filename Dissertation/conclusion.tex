\chapter*{Заключение}                       % Заголовок
\addcontentsline{toc}{chapter}{Заключение}  % Добавляем его в оглавление

%% Согласно ГОСТ Р 7.0.11-2011:
%% 5.3.3 В заключении диссертации излагают итоги выполненного исследования, рекомендации, перспективы дальнейшей разработки темы.
%% 9.2.3 В заключении автореферата диссертации излагают итоги данного исследования, рекомендации и перспективы дальнейшей разработки темы.
%% Поэтому имеет смысл сделать эту часть общей и загрузить из одного файла в автореферат и в диссертацию:

Основные результаты работы заключаются в следующем:
%% Согласно ГОСТ Р 7.0.11-2011:
%% 5.3.3 В заключении диссертации излагают итоги выполненного исследования, рекомендации, перспективы дальнейшей разработки темы.
%% 9.2.3 В заключении автореферата диссертации излагают итоги данного исследования, рекомендации и перспективы дальнейшей разработки темы.
\begin{enumerate}
  \item Разработана математическая модель движения безвинтового подводного робота в форме эллипсоида в жидкости за счет внутреннего гиростатического момента в рамках теории идеальной жидкости. Исследования управляемости модели показали, что для полной управляемости форму робота необходимо сделать винтовой.
  
  \item Разработаны конструкция и экспериментальный образец безвинтового подводного робота в форме эллипсоида. Учтено требование в необходимости винтовой формы робота, к оболочке в виде эллипсоида вращения добавлены винтовые лопасти. Проработана конструкция внутренних компонентов с учетом ограничений в размерах. Разработана система управления роботом.
  
  \item Проведены экспериментальные исследования безвинтового подводного робота в форме эллипсоида, проанализированы результаты. Для более точного описания движения необходимо учитывать в модели вязкое сопротивление, генерацию вихревых структур. В рамках дальнейшего исследования рассматривается движение объекта в форме симметричного профиля NACA 0040 по поверхности жидкости.
  
  \item Разработана математическая модель движения безвинтового надводного робота с оболочкой, имеющей форму симметричного профиля NACA 0040 и острую кромку, в жидкости за счет внутреннего гиростатического момента с учетом вязкого сопротивления. На основе анализа математической модели разработан алгоритм управления роботом, определена форма управляющего воздействия для движения вдоль прямой и вдоль окружности.
  
  \item Разработаны конструкция и экспериментальный образец безвинтового надводного робота с острой кромкой. Разработана система управления роботом.
  
  \item Проведены экспериментальные исследования с безвинтовым надводным роботом с острой кромкой, проанализированы результаты. Показано, что разработанная математическая модель качественно описывает движение робота. Рассмотрены движения вдоль прямой и вдоль окружности при различных управляющих воздействиях.
  
  \item По разработанным конструкциям получен патент на полезную модель безвинтового подводного робота в форме эллипсоида, для разработанных программных продуктов получены свидетельства о регистрации программ ЭВМ.
\end{enumerate}


\textbf{Участие в конференциях}
\begin{itemize}
	\item IV Всероссийская научно-техническая конференция аспирантов, магистрантов и молодых ученых с международным участием «Молодые ученые -- ускорению научно-технического прогресса в XXI веке». (Ижевск, 2016).
	\item Шестая международная конференция «Geometry, Dynamics, Integrable Systems -- GDIS 2016» (Ижевск, 2016 г.)
	\item Машиноведение и инновации. Конференция молодых учёных и студентов (МИКМУС-2018) (Москва, 2018 г.)
	\item International Conference "Scientific Heritage of Sergey A. Chaplygin: nonholonomic mechanics, vortex structures and hydrodynamics" (Чебоксары, 2019 г.)
	\item 30-я международная научно-техническая конференция "Экстремальная робототехника-2019" (Санкт-Петербург, 2019 г.)	
\end{itemize}

\textbf{Публикации}
\begin{itemize}
	
	\item Ветчанин Е. В, Караваев Ю.Л., Калинкин А.А., Пивоварова Е.Н., Клековкин А.В. Модель безвинтового подводного робота //Вестник Удмуртского университета. Математика. Механика. Компьютерные науки. – 2015. – Т. 25. – №. 4. – С. 544-553. (ВАК)
	
	\item Karavaev Y. L., Kilin A. A., Klekovkin A. V. Experimental investigations of the controlled motion of a screwless underwater robot // Regular and Chaotic Dynamics. – 2016. – Т. 21. – №. 7-8. – С. 918-926 (WoS)
	
	\item Klekovkin A.V., Karavaev Yu.L., Kilin A.A., Mamaev I.S. Control screwless fish-like robot with internal rotor // Extreme Robotics,  2019, Vol.1, no. 1, pp. 220-225 (РИНЦ)
	
	\item Karavaev Y.L., Klekovkin A.V., Mamaev I.S., Tenenev V.A., Vetchanin E.V. A Simple Physical Model for Control of an Propellerless Aquatic Robot. //  Mechanical Systems and Signal Processing, 2020, unpublished.
	
\end{itemize}



\textbf{Патенты}
\begin{itemize}

%	\item \todo{ № 2015615728. Программа для управления безвинтовым надводным роботом // А.В. Борисов, И.С. Мамаев, А.А. Килин, Ю.Л. Караваев, А.В. Клековкин, А.В. Шелухо, А.И. Кленов, Е.В. Ветчанин, В.А. Тененев. Заявитель и патентообладатель – ФБГОУ ВО «ИжГТУ имени М.Т. Калашникова»; Заявка: 2015612643, 07.04.2015, опубл. 22.05.2015}


\item Патент на полезную модель. №172254 РФ. Безвинтовой подводный робот //  А.В. Борисов, И.С. Мамаев, А.А. Килин, А.А. Калинкин, Ю.Л. Караваев, А.В. Клековкин, Е.В. Ветчанин; заявитель и патентообладатель – ФБГОУ ВО «ИжГТУ имени М.Т. Калашникова»; Заявка: 2016144812, 15.11.2016, опубл. 3.07.2017


\item № 2017613219. Программа для управления безвинтовым подводным роботом // А.В. Борисов, И.С. Мамаев, А.А. Килин, Ю.Л. Караваев, А.В. Клековкин. Заявитель и патентообладатель – ФБГОУ ВО «ИжГТУ имени М.Т. Калашникова»; Заявка: 2016662663, 22.11.2016, опубл. 16.03.2017


\item № 2019612284. Программа управления безвинтовым надводным роботом с внутренним ротором // А.В. Борисов, И.С. Мамаев, А.А. Килин, А.В. Клековкин, Ю.Л. Караваев. Заявитель и патентообладатель – ФБГОУ ВО "ИжГТУ имени М.Т. Калашникова"; Заявка: 2019610925, 04.02.2019, опубл. 14.02.2019


\end{itemize}


%И какая-нибудь заключающая фраза.

%Последний параграф может включать благодарности.  В заключение автор выражает благодарность и большую признательность научному руководителю Иванову~И.\:И. за поддержку, помощь, обсуждение результатов и~научное руководство. Также автор благодарит Сидорова~А.\:А. и~Петрова~Б.\:Б. за помощь в~работе с~образцами, Рабиновича~В.\:В. за предоставленные образцы и~обсуждение результатов, Занудятину~Г.\:Г. и авторов шаблона *Russian-Phd-LaTeX-Dissertation-Template* за~помощь в оформлении диссертации. Автор также благодарит много разных людей и~всех, кто сделал настоящую работу автора возможной.
