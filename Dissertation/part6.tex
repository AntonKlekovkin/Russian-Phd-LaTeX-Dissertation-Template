\chapter{Модель водоплавающиего недеформируемого рыбоподобного робота}\label{ch:ch6}

\section{Описание конструкции водоплавающиего недеформируемого рыбоподобного робота}

Робот представляет собой полый объект, в продольном сечении имеющий форму профиля крыла NACA 0040 (см. рисунок \ref{Photo_NACA}) длиной 340 мм, шириной 134 мм. Высота робота 80 мм. Форма профиля крыла NACA 0040 задается функцией 
$ y = \frac{T}{0.2}(a_0\sqrt{x} + a_1x + a_2x^2 + a_3x^3 + a_4x^4 $, где $a_0$=0.2969, $a_1$=-0.126, $a_2$=-0.3516, $a_3$=0.2843, $a_4$=-0.1036, T=0.4. Точки контура профиля были рассчитаны в среде Matlab и импортированы в среду разработки Компас-3Д.	
Корпус изготовлен на 3Д-принтере из PLA-пластика, толщина стенки -- 2 мм. Внутри корпуса закреплен ротор с двигателем таким образом, что центр масс всей системы находится максимально близко к нижней грани робота. В качестве двигателя использовался мотор-редуктор фирмы Pololu с энкодером. Для передачи вращения с двигателя к ротору использовалась пара шестерен с передаточным отношением 3.5:1. Внутри так же располагается элемент питанияи плата с микроконтроллером модели STM32F303K8, управляющим вращением двигателем постоянного тока. Для управления двигателем используется драйвер двигателя постоянного тока VNH3SP30 фирмы STMicroelectronics. Энкодер использовался для определения положения ротора в течение экспериментов. Дифференцируя данные, полученные с энкодера, можно получить угловую скорость вращения ротора.
%Характеристики двигателя: номинальное напряжение питания -- 6 В, передаточное отношение редуктора -- 47:1, момент на валу -- 0.459 Нм, максимальная скорость вращения -- 120 об/мин. 


\begin{figure}[h]
	\centering
	\includegraphics[width=1\linewidth]{Photo_NACA.png}%
	\caption{Безвинтовой рыбоподобный робот}
	\label{Photo_NACA}
\end{figure}

Реальная модель робота имеет следующие характеристики: m = $0.905$ кг; $I_0$ = $0.00844$ кг$\cdot$м2; Ротор изготовлен из алюминия, имеет внешний диаметр 110 мм, высоту 12 мм. Масса ротора $m_r$ = $0.327$ кг; момент инерции ротора $I_r$ = $0.00058$ кг$\cdot$м2. Конструкция робота позволяет смещать центр вращения ротора.

Управление осуществляется с персонального компьютера, для которого было разработано специальное программное обеспечение. Все команды роботу передаются по беспроводному каналу связи, используя Bluetooth.


\section{Описание системы управления водоплавающиего недеформируемого рыбоподобного робота}

Для управления безвинтовым недеформируемым рыбоподобным надводным роботом была разработана система управления, структурная схема которой представлена на рисунке~\ref{ControlSystem}.

\begin{figure}[!h]
	\centering
	\includegraphics[width=0.8\linewidth]{ControlSystem.eps}
	\caption{Структурная схема системы управления безвинтового недеформируемого рыбоподобного надводного робота}
	\label{ControlSystem}
\end{figure}

На схеме $ \omega_{set} $ -- заданная скорость вращения ротора. Блок регулятора скорости представляет собой ПИД-регулятор, который обеспечивает поддержание значения заданной скорости $ \omega_{set} $. На выходе данного блока получаем ШИМ-сигнал необходимой скважности. Коэффициенты ПИД-регулятора подобраны экспериментально. Далее ШИМ-сигнал подается на драйвер двигателя постоянного тока, который его усиливает до необходимого напряжения и подает на обмотки двигателя. В данной работе используется драйвер двигателя постоянного тока VNH3SP30 фирмы STMicroelectronics. На валу двигателя располагается датчик положения вала (инкрементальный энкодер с 48 импульсами на оборот), с помощью которого измеряется угол поворота вала двигателя $ \varphi $. Далее с помощью блока преобразования текущего угла, учитывая передаточные отношения редуктора и пары шестеренок на валу двигателя и роторе, получаем значение $ \hat{\omega}_{set} $ -- фактическую скорость вращения ротора. Полученное значение $ \hat{\omega}_{set} $ учитывается блоком регулятора скорости при расчете управляющих сигналов, идущих на двигатель. При вращении ротора данный алгоритм должен выполняться через промежутки времени $ \Delta t \rightarrow 0 $. Выбранный микроконтроллер имеет максимальную частоту работы 72 МГц, что позволяет выбрать $ \Delta t = 1 $ мс. Значение $ \Delta t $ выбрано экспериментально.

Для реализации управляемого движения безвинтового недеформируемого рыбоподобного надводного робота было разработано программное обеспечение нижнего и верхнего уровня.

В программе нижнего уровня реализованы функции управления двигателем, на котором закреплен ротор: движение по прямой и движение по некоторому радиусу. Программа принимает и обрабатывает команды с верхнего уровня (персональный компьютер, планшет, смартфон) по беспроводному каналу связи Bluetooth. Bluetooth-модуль расположен на плате системы управления и соединен с интерфейсом USART микроконтроллера. Командами задаются значения угловой скорости ротора, время вращения ротора на заданной скорости и время перехода от одной скорости вращения к другой. Реализованы команды начала вращения ротора по установленным параметрам и его остановки.

На двигателе установлен датчик углового перемещения вала – энкодер. Программа считывает с него данные, рассчитывает текущее положение ротора, учитывая передаточное отношение редуктора и сохраняетэти данные в памяти микроконтроллера. По запросу эти данные отправляются на программное обеспечение верхнего уровня. Так же с помощью численного дифференцирования можно получить фактическую угловую скорость и угловое ускорение ротора. Значение фактической угловой скорости используется в программе для поддержания заданной скорости вращения ротора.

Программа нижнего уровня предназначена для отладочной платы STM32-Nucleo32, на борту которой расположен микроконтроллер STM32F303K8T6. Это 32-разрядный микроконтроллер с ядром ARM Cortex-M4, работающий на частоте до 72 МГц.

Программа верхнего уровня разработана для смартфона на операционной системе Android версии не ниже 9 с помощью онлайн-сервиса MIT App Inventor. В данном сервисе используется визуальный язык программирования, который позволяет разработать приложение, используя графический интерфейс. Часть программы, разработанной с помощью данного сервиса представлена на рисунке~\ref{AndroidInventor}.

\begin{figure}[!h]
	\centering
	\includegraphics[width=0.7\linewidth]{AndroidInventor.png}
	\caption{Разработка андроид-приложения в MIT App Inventor}
	\label{AndroidInventor}
\end{figure}

Интерфейс программы представлен на рисунке~\ref{AndroidScreen}.

\begin{figure}[!h]
	\centering
	\includegraphics[width=0.3\linewidth]{AndroidScreen.png}
	\caption{Интерфейс программы управления безвинтовым недеформируемым рыбоподобным надводным роботом}
	\label{AndroidScreen}
\end{figure}

Данная программа позволяет подключится к bluetooth-устройству, установленному на роботе и передавать необходимые команды. Используя поле "Sending data" можно отправить необходимую последовательность байтов роботу. В поле "Receive data" отображаются данные, принятые от робота. Девять кнопок в нижней части экрана позволяют управлять роботом, используя разные режимы движения: запуск и остановка вращения ротора для движения по прямой и по окружности, изменение периода управляющих импульсов.

Длительное нажатие на кнопку позволяет изменить команду, присвоенную кнопке при помощи поля "Buttons Commands".

\section{Отладка системы управления}

\qquad После монтажа элементов на печатную плату системы управления, были проверены цепи питания и "земли" на короткое замыкание. После успешной проверки плата подключалась к блоку питания для мониторинга потребляемого тока. Рабочий ток платы на холостом ходу не должен превышать 50 мА.

После загрузки прошивки в микроконтроллер был проверен канал связи. В режиме отладки необходимо убедиться, что все команды, отправляемые с программы верхнего уровня, доходят без ошибок до программы нижнего уровня; данные с программы нижнего уровня доходят без ошибок до программы верхнего уровня.

После успешной проверки можно установить плату системы управления в корпус робота, подключить двигатель и аккумулятор и еще раз проверить все команды с подключенным двигателем и вращением ротора.


\clearpage