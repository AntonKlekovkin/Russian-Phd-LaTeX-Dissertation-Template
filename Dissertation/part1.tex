\chapter{Анализ механизмов движения мобильных плавающих роботов}\label{ch:ch1}

\section{Введение}\label{sec:ch1/sec1}

\section{Обзор подводных аппаратов}\label{sec:ch1/sec21}


\subsection{История обитаемых подводных аппаратов}\label{subsec:ch1/sec21/sub1}

История подводных аппаратов началась с подводных лодок, которые появились в начале XX века и имели исключительно военное назначение и входили в состав военно-морских флотов. Скрытное передвижение под водой позволяло эффективно их использовать против надводных кораблей. В 1914 году, в Германии, была спущена на воду первая исследовательская подводная лодка «Лолиго», но начавшаяся Первая мировая война не позволила данному аппарату заниматься мирными задачами. 

Следующей исследовательской подводной лодкой был американский «Наутилус». Спущен на воду в 1917, но только в 1931 году переоборудован в исследовательскую подлодку. В торпедном отсеке была оборудована шлюзовая камера для выхода водолазов и работы с океанологической лебедкой. В других отсеках установлено дополнительное исследовательское и навигационное оборудование, в том числе эхолот и гирокомпас.
Недостатком подводных лодок является небольшая 	глубина погружения. В первой половине XX века лодки погружались на глубину до нескольких десятков метров. Современные атомные подводные лодки могут спускаться на глубину до 600 метров (абсолютный рекорд погружения поставила советская атомная подводная лодка «Комсомолец» – 1027 метров, 1985 г.). Такой глубины для исследовательских целей было недостаточно, поэтому исследовательские подводные лодки использовались, в основном, как базы для подводных биологических исследований.

История глубоководных погружений в подводных аппаратах начинается с 1930-х годов с экспериментов исследователей Мирового океана У. Биба и О. Бартона [?]. В качестве конструкции подводного аппарата они выбрали форму сфероида, который должен был погружаться на глубину на стальном тросе, который был прикреплен к основному судну. В 1948 году разработанная батисфера позволила достигнуть глубины 1360 метров. Но данная конструкция имела серьезный недостаток – тот самый стальной трос, который буквально «привязывал» батисферу к судну и не позволял достичь большей глубины. А также требовались значительные финансовые затраты на содержание специального судна.
Учитывая недостатки существующих на тот момент подводных аппаратов, профессор О. Пиккар в 1950-х годах разработал новую конструкцию и специальное оборудование глубоководного подводного аппарата – батискафа. Экипаж батискафа размещался в герметичной сфере – гондоле, имеющей благодаря своим толстым стальным стенкам и весу находящегося в ней оборудования отрицательную плавучесть. Необходимая же плавучесть батискафа обеспечивалась за счет поплавка, заполненного бензином. Батискаф Пиккара совершил самые глубоководные в мире погружения, его рекорды не превзойдены до сих пор. И хотя большие габариты и особенности конструкции этого аппарата осложняют его эксплуатацию, батискаф до сих пор остается единственным аппаратом, способным доставить исследователей в любую точку океанского дна.

Примерно в это же время Жак-Ив Кусто разработал свой подводный аппарат – «ныряющее блюдце» []. Прочный корпус "блюдца" обладал положительной плавучестью. Глубина его погружения, правда, не превышала 300 м (позже она увеличилась до 350 м), но зато аппарат имел хорошую маневренность, небольшой вес и малые габариты.

Дальнейшему прогрессу в создании подводных обитаемых аппаратов способствовало появление новых материалов на основе эпоксидных смол с наполнителем из стеклянных шариков. Материалы эти имеют малый удельный вес, но, в то же время, обладают высокой прочностью. Применение их позволило резко снизить вес корпуса подводных аппаратов.

\subsection{История необитаемых подводных аппаратов}\label{subsec:ch1/sec21/sub2}

За последние годы в различных странах, занимающихся морскими технологиями, было создано более 9 тысяч самоходных необитаемых подводных аппаратов (НПА) для широкого круга задач таких как: аварийно-спасательные, обзорно-поисковые, научно-исследовательские, экологический мониторинг и другие виды работ.

Современные самоходные НПА представляют собой отдельный класс робототехнических объектов. Они имеют различные размеры и массу: от нескольких килограмм до 5 тонн и более. Отдельные модели могут погружаться на глубину до 7000 метров. При всем их разнообразии общепризнанной классификации в этом разделе робототехники еще не сложилось. Однако, все НПА можно разделить на 2 больших подкласса: неавтономные и автономные НПА.

К неавтономным НПА можно отнести буксируемые и самоходные привязные подводные аппараты. Самоходные аппараты в свою очередь могут быть плавающими, донными или с комбинированным типом движения [38]. Такие недостатки как: зависимость от основного судна, ограниченный радиус действия, наличие устройства управления натяжением кабеля-связки, а так же достижения в области энергетики, электроники и информационных технологий послужили стимулом к развитию автономных НПА.

К автономным НПА (АНПА) относятся самоходные НПА с автономной системой энергообеспечения и беспроводным каналом связи. Так же существуют полуавтономные НПА, которые имеют автономную систему энергообеспечения и проводной канал управления и связи [39]. Автономные и полуавтономные НПА можно классифицировать по целевому назначению (военное, гражданское, исследовательское), массогабаритным характеристикам (микро, мини, малые, средние, большие), конструктивному облику (классическая гидродинамическая форма, планерная форма, плоская форма для солнечный панелей, бионическая форма).

За рубежом активное развитие АНПА пришлось на 90-е годы XX века [12, 14, p7]. За несколько лет было разработано около 30 совершенно новых моделей АНПА по всему миру. Мировыми лидерами в разработке подобных систем были США [15,17, 28], Великобритания [18, 19], Канада [20], Япония [16] и другие страны. Усовершенствование моделей АНПА, а также новые разработки ведутся до настоящего времени [13, p1, p3, p5, p6, p8]. За последние годы появлялось в среднем около 70 новых проектов АНПА ежегодно [38].

Для автономных НПА помимо конструкции важно разработать адекватную модель движения и систему управления. Навигационная система и система планирования траектории должны определять местоположение аппарата и движение по выбранному курсу [21, p2, 22]. При наличии датчиков или системы технического зрения нужно предусмотреть сохранение получаемой информации и дальнейшую ее обработку [26, 27, 29, 30]. В системах управления могут использоваться системы нечеткой логики и нейронные сети [23, 24, 25].

В России в настоящее время одним из ведущих институтов работающих по данному направлению является Институт проблем морских технологий Дальневосточного отделения Российской академии наук (ИПМТ ДВО РАН) [1,2,4]. Сотрудниками института разрабатываются конструкции моделей АНПА [5, 11], изучаются динамические характеристики [1], бортовые системы управления [9], алгоритмы движения [10].

Из вышеприведенного обзора можно заметить, что большинство подводных аппаратов для перемещения используют вращающийся винт. Часть АНПА разрабатывается на основе бионических принципов и для перемещения используются принципы движения живых морских существ (например, плавники рыб, медузы, змеи). Но существует ряд задач, в которых нельзя активно вмешиваться в состояние среды. В таких случаях целесообразно использовать внутренние силы, например, движение робота за счет перемещения внутренних масс, встроенных в корпус робота [31,32]. Управление движением робота с перемещающимися внутренними массами в жидкой среде описано в работах [33, 34, 35]. В статьях разработана математическая модель, проведены численные исследования, приведены результаты моделирования движения. В работе [36] рассмотрена трехмерная вязкая гидродинамика движения тела с переменным распределением массы. Существует патент на полезную модель безвинтового надводного робота [37]. Подводного варианта робота не представлено.


\section{Движение в жидкости за счет использования гребных винтов}\label{sec:ch1/sec2}


\section{Движение в жидкости за счет изменения формы тела}\label{sec:ch1/sec3}


\section{Движение в жидкости за счет реактивной тяги}\label{sec:ch1/sec4}



\section{Движение в жидкости за счет внутренних механизмов}\label{sec:ch1/sec5}



\subsection{Движение в жидкости за счет изменения положения центра масс системы}\label{subsec:ch1/sec5/sub1}

В работе \cite{Volkova_Jatsun} рассматривается робот, состоящий из корпуса и двух по-движных внутренних масс, которые перемещаются относительно корпуса по прямолинейным направляющим. Взаимодействие робота со средой осуществляется только за счет четырех опорных поплавков с изменяемым углом наклона относительно вертикали. Движение происходит за счет изменения силы трения вдоль продольной оси корпуса при повороте поплавков. В той же работе приведена математическая модель и дается численное моделирование, позволившее изучить управляемые движения робота на примере прямолинейного и вращательного движения.

Известна модель безвинтового надводного робота, приводящегося в движение путем смещения центра масс двух эксцентриков, приводимых в движение одним электродвигателем \cite{patent_BNR}.

В работе \cite{Ramodanov_Tenenev} рассматривается задача о движении тела в вязкой жидкости, за счет перемещения внутренних масс, при котором внешняя оболочка тела остается неизменной. Приведена математическая модель, построенная на гидродинамических уравнениях Навье-Стокса. В результате численного моделирования показано существенное влияние сил и момента вязкого сопротивления на траекторию движения, выявлены отличия движения тела в вязкой жидкости по сравнению с идеальной. На основе полученных результатов в работе \cite{Vetchanin_Tenenev_2011} решена задача оптимального управления движением тела по заданной траектории за счет перемещения внутренних масс, с применением гибридного генетического алгоритма. В результате получены аппроксимационные зависимости для сил, действующих на тело.

Исследование характеристик движения тела с переменным распределением массы в трехмерной вязкой жидкости проведено в работе \cite{Vetchanin_Mamaev_Tenenev_ND_2012}, а в работе \cite{Kilin_Vetchanin_DAN_2016} рассмотрено управляемое движение при наличии циркуляции вокруг тела. В этих работах показана возможность перемещения тела в произвольном направлении, а также возможность преодоления силы тяжести телом с плавучестью близкой к нулевой.

\subsection{Движение в жидкости за счет изменения гиростатического момента системы}\label{subsec:ch1/sec5/sub2}



