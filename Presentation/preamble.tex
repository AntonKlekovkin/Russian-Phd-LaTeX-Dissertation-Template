\begin{frame}[noframenumbering,plain]
    \setcounter{framenumber}{1}
    \maketitle
\end{frame}

%[shrink=5]

\begin{frame}
\frametitle{Цель и задачи работы}
\footnotesize

\quad \textbf{Целью} данной работы является исследование механизмов, обеспечивающих движение водоплавающих роботов за счет изменения гиростатического момента, возникающего от вращения роторов, расположенных внутри оболочки объектов.

\textbf{Задачи}:
\begin{itemize}
	\item Построение математической модели движения водных роботов за счет изменения внутреннего гиростатического момента: безвинтового подводного робота с внутренними роторами в жидкости и недеформируемого водного робота с острой кромкой.
	\item Разработка алгоритма управления для реализации движения водных роботов: безвинтового подводного робота с внутренними роторами и недеформируемого водного робота с острой кромкой.
	\item Разработка конструкции и прототипов водных роботов: безвинтового подводного робота с внутренними роторами и недеформируемого водного робота с острой кромкой; разработка систем управления.
	\item Проведение натурных экспериментов и исследования влияния режимов работы механизма на динамику роботов: безвинтового подводного робота с внутренними роторами и недеформируемого водного робота с острой кромкой.
	\item Сравнение экспериментальных данных с результатами численного моделирования.
\end{itemize}
\end{frame}

\begin{frame}
    \frametitle{Положения, выносимые на защиту}
\small    
    \begin{itemize}
        \item Математическая модель движения безвинтового подводного робота с внутренними роторами в жидкости за счет изменения внутреннего кинетического момента.
        \item Математическая модель движения недеформируемого водного робота с острой кромкой в жидкости за счет изменения внутреннего кинетического момента с учетом вязкого трения.
        \item Алгоритм управления безвинтовым подводным роботом в форме эллипсоида в жидкости за счет изменения внутреннего гиростатического момента.
        \item Алгоритм управления недеформируемым водным роботом с острой кромкой в жидкости за счет изменения внутреннего гиростатического момента.
        \item Конструкция роботов, реализующих движение в жидкости за счет изменения внутреннего гиростатического момента:  безвинтового подводного робота с внутренними роторами и недеформируемого водного робота с острой кромкой.
        \item Результаты экспериментальных исследований по оценке разработанных алгоритмов управления для безвинтового подводного робота с внутренними роторами и недеформируемого водного робота с острой кромкой.
    \end{itemize}
\end{frame}
%\note{
%    Проговариваются вслух положения, выносимые на защиту
%}
%
%\begin{frame}
%    \frametitle{Содержание}
%    \tableofcontents
%\end{frame}
%\note{
%    Работа состоит из четырёх глав.
%
%    \medskip
%    В первой главе \dots
%
%    Во второй главе \dots
%
%    Третья глава посвящена \dots
%
%    В четвёртой главе \dots
%}
