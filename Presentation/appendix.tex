%\begin{frame}
%    \frametitle{Ответы на замечания ведущей организации НИИ~<<Рога~и~копыта>>}
%    \begin{itemize}
%        \item Замечание -- ответ
%        \item Замечание -- ответ
%        \item Замечание -- ответ
%        \item Замечание -- ответ
%        \item Замечание -- ответ
%    \end{itemize}
%\end{frame}
%
%\begin{frame}
%    \frametitle{Ответы на замечания оф. оппонента Иванова\,И.\,И}
%    \begin{itemize}
%        \item Замечание -- ответ
%        \item Замечание -- ответ
%        \item Замечание -- ответ
%        \item Замечание -- ответ
%        \item Замечание -- ответ
%    \end{itemize}
%\end{frame}
%
%\begin{frame}
%    \frametitle{Ответы на замечания Петрова\,П.\,П}
%    \begin{itemize}
%        \item Замечание -- ответ
%        \item Замечание -- ответ
%        \item Замечание -- ответ
%        \item Замечание -- ответ
%        \item Замечание -- ответ
%    \end{itemize}
%\end{frame}

\begin{frame}
\frametitle{Участие в конференциях}
\begin{itemize}
	\item IV Всероссийская научно-техническая конференция аспирантов, магистрантов и молодых ученых с международным участием «Молодые ученые -- ускорению научно-технического прогресса в XXI веке». (Ижевск, 2016).
	\item Шестая международная конференция «Geometry, Dynamics, Integrable Systems -- GDIS 2016» (Ижевск, 2016 г.)
	\item Машиноведение и инновации. Конференция молодых учёных и студентов (МИКМУС-2018) (Москва, 2018 г.)
	\item International Conference "Scientific Heritage of Sergey A. Chaplygin: nonholonomic mechanics, vortex structures and hydrodynamics" (Чебоксары, 2019 г.)
	\item 30-я международная научно-техническая конференция "Экстремальная робототехника-2019" (Санкт-Петербург, 2019 г.)	
\end{itemize}
\end{frame}

\begin{frame}
\frametitle{Публикации}
\begin{itemize}

\item Ветчанин Е. В, Караваев Ю.Л., Калинкин А.А., Пивоварова Е.Н., Клековкин А.В. Модель безвинтового подводного робота //Вестник Удмуртского университета. Математика. Механика. Компьютерные науки. – 2015. – Т. 25. – №. 4. – С. 544-553. (ВАК)

\item Karavaev Y. L., Kilin A. A., Klekovkin A. V. Experimental investigations of the controlled motion of a screwless underwater robot // Regular and Chaotic Dynamics. – 2016. – Т. 21. – №. 7-8. – С. 918-926 (WoS)

\item Klekovkin A.V., Karavaev Yu.L., Kilin A.A., Mamaev I.S. Control screwless fish-like robot with internal rotor // Extreme Robotics,  2019, Vol.1, no. 1, pp. 220-225 (РИНЦ)

\item Yury Karavaev, Anton Klekovkin, Ivan Mamaev, Valentin Tenenev, Eugene Vetchanin, "A Simple Physical Model for Control of an Propellerless Aquatic Robot",  Mechanical Systems and Signal Processing, 2020, unpublished. (WoS)

\end{itemize}
\end{frame}

\begin{frame}
\frametitle{Патенты}
\begin{itemize}

%	\item № 2015615728. Программа для управления безвинтовым надводным роботом // А.В. Борисов, И.С. Мамаев, А.А. Килин, Ю.Л. Караваев, А.В. Клековкин, А.В. Шелухо, А.И. Кленов, Е.В. Ветчанин, В.А. Тененев. Заявитель и патентообладатель – ФБГОУ ВО «ИжГТУ имени М.Т. Калашникова»; Заявка: 2015612643, 07.04.2015, опубл. 22.05.2015


\item Патент на полезную модель. №172254 РФ. Безвинтовой подводный робот //  А.В. Борисов, И.С. Мамаев, А.А. Килин, А.А. Калинкин, Ю.Л. Караваев, А.В. Клековкин, Е.В. Ветчанин; заявитель и патентообладатель – ФБГОУ ВО «ИжГТУ имени М.Т. Калашникова»; Заявка: 2016144812, 15.11.2016, опубл. 3.07.2017


\item № 2017613219. Программа для управления безвинтовым подводным роботом // А.В. Борисов, И.С. Мамаев, А.А. Килин, Ю.Л. Караваев, А.В. Клековкин. Заявитель и патентообладатель – ФБГОУ ВО «ИжГТУ имени М.Т. Калашникова»; Заявка: 2016662663, 22.11.2016, опубл. 16.03.2017


\item № 2019612284. Программа управления безвинтовым надводным роботом с внутренним ротором // А.В. Борисов, И.С. Мамаев, А.А. Килин, А.В. Клековкин, Ю.Л. Караваев. Заявитель и патентообладатель – ФБГОУ ВО "ИжГТУ имени М.Т. Калашникова"; Заявка: 2019610925, 04.02.2019, опубл. 14.02.2019


\end{itemize}
\end{frame}

\begin{frame}
\frametitle{Кинетическая энергия}

\begin{minipage}{0.47\linewidth}
	Кинетическая энергия оболочки:
	\begin{gather}
	T_s = \frac{1}{2} m_s  \bigl( \bV,\, \bV \bigr) + \frac{1}{2} \bigl( {\bm I}_s {\bOm},\, {\bOm} \bigr),\nonumber
	\end{gather}
\end{minipage}
\hfill
\begin{minipage}{0.47\linewidth}
	Кинетическая энергия жидкости:
	\begin{gather}
	T_f = \frac{1}{2} \bigl( \bLam_1 \bV,\, \bV \bigr) + \frac{1}{2} \bigl( {\bLam} _2 {\bOm},\, {\bOm} \bigr).\nonumber
	\end{gather}
\end{minipage}

\vspace{2mm}

Кинетическая энергия $k$-го ротора:
\begin{gather}
T_k = \frac{1}{2} m_R \bigl( \bV + {\bOm} \times \br_k, \bV + {\bOm} \times \br_k \bigr) + \frac{1}{2}\Bigl({\bm I}_k \bigl( {\bOm} + \omega_k \bn_k \bigr), {\bOm} + \omega_k \bn_k \Bigr),\nonumber
\end{gather}

Суммарная кинетическая энергия всей системы: 		
\footnotesize
\begin{gather*}
\begin{split}
T = T_f + T_s + \sum _{k=1}^3 T_k = \frac{1}{2} \bigl( {\bbI} {\bOm},\, {\bOm} \bigr) + \bigl( {\bbB} {\bOm},\, \bV \bigr) + \frac{1}{2} \bigl( {\bbC} \bV,\, \bV \bigr) + \bigl( {\bOm},\, \bK(t) \bigr) + \frac{1}{2} \sum_{k=1}^3 i \omega_k^2 (t),
\end{split}
\end{gather*}

\small
Матрицы $\bbI$, $\bbB$, $\bbC$ имеют вид:	

\begin{minipage}{0.57\linewidth}
	\vspace{-3mm}
	\begin{gather}
	{\bbI} = {\bLam}_2 + {\bbI}_s + \sum _{k=1}^3 {\bbI}_k + \frac{1}{2} m_R \sum _{k=1}^3 \bigl( \br_k^2{\bbE} - \br_k \otimes \br_k \bigr),\nonumber \\
	{\bbC} = m {\bbE} + {\bLam}_1,\nonumber \quad	 m = m_s + 3 m_R		\nonumber
	\end{gather}
\end{minipage}
\hfill
\begin{minipage}{0.4\linewidth}
	\vspace{-3mm}
	\begin{gather}
	{\bbB} = m \begin{pmatrix}\nonumber
	0 & z_c & -y_c\\
	-z_c & 0 & x_c\\
	y_c & -x_c & 0
	\end{pmatrix}\nonumber	,	
	\end{gather}
\end{minipage}	

\vspace{2mm}

где $x_c$, $y_c$, $z_c$ --- компоненты радиус-вектора $\br_c$ центра масс системы. 

$\bK(t)=\sum \limits_{k=0}^3 i \omega_k (t)\bn_k$ --- вектор гиростатического момента. 


\end{frame}

\begin{frame}[shrink=5]
\frametitle{Исследование управляемости системы}

Для исследования управляемости представим систему уравнений движения в виде

{\small \begin{gather*}
\dot{\bq} = \bX_1 (\psi,\, \theta,\, \varphi) \Omega_1 + \bX_2 (\psi,\, \theta,\, \varphi) \Omega_2 + \bX_3 (\psi,\, \theta,\, \varphi) \Omega_3,\\
\begin{gathered}
\bX_1 = \left( \dfrac{\sin \varphi}{\sin \theta} ,\, \cos \varphi ,\, -\cot\theta \sin\varphi ,\, 
\dfrac{ m y_c\alpha_3 }{c_3} - \dfrac{m z_c \alpha_2}{c_2}, \, 
\dfrac{m y_c \beta_3 }{c_3} - \dfrac{m z_c \beta_2}{c_2} ,\, 
\dfrac{m y_c \gamma_3 }{c_3} - \dfrac{m z_c \gamma_2}{c_2} \right)^T,\\
\bX_2 = \left( \dfrac{\cos \varphi}{\sin \theta} ,\, -\sin \varphi ,\, -\cot\theta \cos\varphi ,\, 
\dfrac{m z_c\alpha_1 }{c_1} - \dfrac{m x_c \alpha_3}{c_3}, \, 
\dfrac{m z_c \beta_1 }{c_1} - \dfrac{m x_c \beta_3}{c_3} ,\, 
\dfrac{m z_c \gamma_1 }{c_1} - \dfrac{m x_c \gamma_3}{c_3} \right)^T,\\
\bX_3 = \left( 0 ,\, 0 ,\, 1 ,\, 
\dfrac{m x_c \alpha_2 }{c_2} - \dfrac{m y_c \alpha_1}{c_1}, \, 
\dfrac{m x_c \beta_2 }{c_2} - \dfrac{m y_c \beta_1}{c_1} ,\, 
\dfrac{m x_c \gamma_2 }{c_2} - \dfrac{m y_c \gamma_1}{c_1} \right)^T,
\end{gathered}
\end{gather*}}
где $\bq = (x,\, y,\, z \, \psi,\, \theta,\, \varphi)$ -- вектор обобщенных координат.



Скобка Ли для векторных полей $ \bbs v $ и $ \bbs u $ имеет выражение
\begin{gather*}
[\bbs v, \bbs u]_{i}=\sum_{j}v_{j}\frac{\partial u_{i}}{\partial q_{j}}-u_{j}\frac{\partial v_{i}}{\partial q_{j}}
\end{gather*}


\end{frame}

\begin{frame}
\frametitle{Методика определения коэффициентов}

Уравнения Навье-Стокса, записанные относительно криволинейной системы координат $(\xi,\, \eta)$, связанной с движущимся профилем имеют вид:
\begin{gather*}
\begin{gathered}
\pder{Du_1}{\xi} + \pder{Du_2}{\eta} = 0,\\
\begin{split}
\pder{u_1}{t} + \frac{1}{D^2} \biggl( \pder{D(u_1 - w_1)u_1}{\xi} + & \pder{D(u_2 - w_2)u_1}{\eta} \biggr) = \\
= - \frac{1}{D\rho} \pder{p}{\xi} + & \frac{\nu}{D^2} \left( \pdder{u_1}{\xi}{2} + \pdder{u_1}{\eta}{2} \right) + \beta_1 + 2 u_2 \omega
\end{split}\\
\begin{split}
\pder{u_2}{t} + \frac{1}{D^2} \biggl( \pder{D(u_1 - w_1)u_2}{\xi} + & \pder{D(u_2 - w_2)u_2}{\eta} \biggr) = \\
= - \frac{1}{D\rho} \pder{p}{\eta} + & \frac{1}{D^2} \left( \pdder{u_2}{\xi}{2} + \pdder{u_2}{\eta}{2} \right) + \beta_2 - 2 u_1 \omega,
\end{split}
\end{gathered}
\end{gather*}
где $u_1$, $u_2$ --- проекции вектора скорости жидкости на криволинейные оси, $p$ --- давление, $\rho$ --- плотность жидкости, $\nu$ --- кинематическая вязкость, $w_1 = v_1 - \omega x_2(\xi,\, \eta)$, $w_2 = v_2 + \omega x_1(\xi,\, \eta)$ --- компоненты переносной скорости. 

\end{frame}

\begin{frame}
\frametitle{Методика определения коэффициентов}


Коэффициент Ламэ $D$ и члены $\beta_1$, $\beta_2$, возникающие вследствие искривления сеточных линий, имеют вид:
\begin{gather*}
D = \sqrt{\left( \pder{x_1}{\xi}\right)^2 + \left( \pder{x_2}{\xi}\right)^2  } =\sqrt{\left( \pder{x_1}{\eta}\right)^2 + \left( \pder{x_2}{\eta}\right)^2  },\\
\begin{split}
\beta_1 = \frac{\nu}{D^3} \Biggl( u_1\left( \pdder{D}{\xi}{2} + \pdder{D}{\eta}{2} \right) + {} & {} 2 \pder{u_1}{\xi} \pder{D}{\xi} + 2 \pder{u_2}{\xi} \pder{D}{\eta} + \\
+ {} & {} \frac{2u_2}{D} \pder{D}{\xi} \pder{D}{\eta} - \frac{2u_1}{D} \left( \pder{D}{\eta} \right)^2  \Biggr)
\end{split}\\
\begin{split}
\beta_2 = \frac{\nu}{D^3} \Biggl( u_2\left( \pdder{D}{\xi}{2} + \pdder{D}{\eta}{2} \right) + {} & {} 2 \pder{u_1}{\eta} \pder{D}{\xi} + 2 \pder{u_2}{\eta} \pder{D}{\eta} + \\
+ {} & {} \frac{2u_1}{D} \pder{D}{\xi} \pder{D}{\eta} - \frac{2u_2}{D} \left( \pder{D}{\xi}\right)^2 \Biggr)
\end{split}
\end{gather*}

\end{frame}




\begin{frame}
\frametitle{Методика определения коэффициентов}

При известных распределениях $u_1$, $u_2$, $p$ cилы $f_1$, $f_2$ и момент $g$, действующие на тело со стороны жидкости, определяются следующими интегралами по контуру $L$ профиля:
\begin{gather*}
\begin{gathered}
f_1 = \oint_L \left( p \pder{x_2}{\xi} + \frac{\rho \nu}{D} \pder{u_1}{\eta} \pder{x_1}{\xi} \right) d\xi,\\
f_2 = \oint_L \left( -p \pder{x_1}{\xi} + \frac{\rho \nu}{D} \pder{u_1}{\eta} \pder{x_2}{\xi} \right) d\xi,\\
g = \oint_L \left( x_1 \left( -p \pder{x_1}{\xi} + \frac{\rho \nu}{D} \pder{u_1}{\eta} \pder{x_2}{\xi} \right) - x_2 \left( p \pder{x_2}{\xi} + \frac{\rho \nu}{D} \pder{u_1}{\eta} \pder{x_1}{\xi} \right) \right) d\xi - \dot{k}(t).
\end{gathered}\label{eq.fgNS}
\end{gather*}

Поскольку движение робота является существенно нестационарным, оказывается невозможным определение коэффициентов $\lambda_{11}$, $\lambda_{22}$, $\lambda_{33}$, $\lambda_{23}$, $c_1$, $c_2$, $c_3$ по-отдельности. Таким образом данные коэффициенты должны определяться совместно с учетом нестационарности движение профиля.

%	Для задания граничных условий, соответствующих нестационарному движению профиля, будем использовать экспериментальные данные для прототипа, описанного в разделе 1 с периодом управляющего воздействия $T = 1$ с, которые представляют собой таблицу значений:
%	\begin{gather}
%	(t_i,\, x_i,\, y_i,\, \alpha_i),\quad i = 0,\, \ldots,\, N.\label{eq.expXYPhi}
%	\end{gather}
%	Здесь $t_i$ --- момент времени, $(x_i,\, y_i)$ --- положение центра масс профиля в момент времени $t_i$, $\alpha_i$ --- ориентация профиля в момент времени $t_i$.
%	
\begin{gather*}
\begin{gathered}
\lambda_{11}^{(1)} \approx 0.3087, \quad 
\lambda_{22}^{(1)} \approx -0.5796,\quad 
\lambda_{23}^{(1)} \approx 0.039085,\\
\lambda_{22}^{(2)} \approx 2.0996,\quad 
\lambda_{23}^{(2)} \approx 0.17629,\quad
\lambda_{11}^{(2)} \approx -7.9826,\\
\lambda_{23,l}^{(3)} \approx 0.083474,\quad
\lambda_{33}^{(2)} \approx 0.018935,\quad
\lambda_{11}^{(3)} - \lambda_{22}^{(3)} \approx - 4.7550,\quad
\lambda_{23,r}^{(3)} \approx 1.4488,\\
c_1 = 0.04715,\quad c_2 = 17.702,\quad c_3 = 0.092872.
\end{gathered}\label{eq.coeffs2}
\end{gather*}


\end{frame}